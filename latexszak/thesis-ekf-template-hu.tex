\documentclass[colorlinks]{thesis-ekf}
\usepackage[T1]{fontenc}
\usepackage[utf8]{inputenc}
\PassOptionsToPackage{defaults=hu-min}{magyar.ldf}
\usepackage[magyar]{babel}
\usepackage{graphicx,amsmath,amssymb,amsthm}
\graphicspath{{./figures/}} % A képfájlokat a [figures] mappába kell tenni!
\footnotestyle{rule=fourth}

\newtheorem{tetel}{Tétel}[chapter]
\newtheorem{lemma}[tetel]{Lemma}
\theoremstyle{definition}
\newtheorem{definicio}[tetel]{Definíció}
\newtheorem{feladat}[tetel]{Feladat}
\theoremstyle{remark}
\newtheorem{megjegyzes}[tetel]{Megjegyzés}
\newtheorem*{megoldas}{Megoldás}

\begin{document}
\logo{\includegraphics[width=9cm]{eke-logo}}
\institute{Matematikai és Informatikai Intézet}
\title{A szakdolgozat címe}
\author{Szerző neve\\ szak}
\supervisor{Tanár neve\\ beosztás}
\city{Eger}
\date{2017}
\maketitle
\tableofcontents

\chapter*{Előszó}
Az alábbi szakdolgozatot az Eszterházy Károly Egyetemen a Programtervező Informatikus BSC szak elvégzése céljából készítettem. Fő témája az adatbázis-kezelő rendszerek bemutatása és felhasználási területei. Ezt három kategóriába soroltam, melyek leginkább meghatározóak az adatbáziskezelés világában. Ezek a következőek: SQL, NoSQL, BigData. Ezen csoportok alatt rengeteg féle megvalósítási mód, motor, rendszer jött létre és terjedt el. Például a MSSQL, Oracle RDBMS, MongoDB, MySQL, PostgreSQL, felhő alapú szolgáltatások. Hogy miért is írok erről?\\
Manapság már elengedhetetlen dolog az életünkben a számítógépes világ és az internet. Ha csak az időjárást vagy a holnapi TV műsort szeretnénk megnézni, akkor sokszor az első dolgunk: kedvenc böngészőnkbe begépelni a kívánt weboldalt és már elénk is tárul a végtelen mennyiségű információ. És még is mi áll e mögött? Minek köszönhető, hogy bármit megtalálhatunk? Ha nem csak egy statikus weboldalról van szó, akkor elengedhetetlen egy adatbázis az adatok tárolására.\\
Régen mindez papíralapú volt. Könyvtárak végtelen polcai, jegyzetek, űrlapok fehér oldalai, napilapok ezrei jelentek meg világszerte. Az irodákban omladoztak a polcok a nagyobbnál-nagyobb dossziéktól, emberek órákon át keresték a rég elfeledett papírfecnin lévő utolsó telefonszámot, számlázási címet. Ezt rohamos tempóban váltja fel az elektronikus alapú tárolás.\\
Mint a papíralapú adatokat is, ezeket is több formátumban, rendszerben lehet tárolni és onnan visszaolvasni. Egy otthoni, 100 kilobájtos Excel táblázaton át nagyvállalatok több száz terrabájtos BigData rendszeréig bezárólag. Viszont ennek köszönhetően az adatok közötti keresésnek az ideje töredéke lett, a rendszerezés egyszerűbbé vált, egy adatbázis pedig elfér akár egy 15 milliméteres MicroSD kártyán is.\\
Számomra nagyon érdekes az, hogy egy azonos célú technológiát milyen színesen, mennyi féle módon lehetséges megvalósítani. Célom az, hogy bemutassam ezen módszerek főbb tulajdonságait, a jó és rossz oldalukat egyaránt. Igyekszem párhuzamot vonni a különböző technológiák, megvalósítások között. Mit mi váltott fel az idő során, melyiket mire használják/ták, s a történelem, illetve napjaink folyamán mely érdekes tények jellemezték és jellemzik ezeket a rendszereket.

\chapter{Fejezet címe}
\section{Szakasz címe}
\subsection{Alszakasz címe}

\begin{thebibliography}{1}
\bibitem{cimke} \textsc{Szerző}: Cím, Kiadó, Hely, évszám.
\end{thebibliography}
\end{document}
